\part{Organisatorisches}
\section{1. Vorlesung}
\begin{itemize}
	\item Dozenten
		\begin{itemize}
			\item  Reinefeld
			\item Schintke (schintke@zib.de)
		\end{itemize}
	\item Stil
		\begin{itemize}
			\item VL (2-3 h)
			\item Übungen (1-2 h)
			\item Projektarbeit
		\end{itemize}
	\item Voraussetzungen
		\begin{itemize}
			\item Grundstudium: Programmiesprachen, OS, Rechnerarchitektur, nichtsequentielle Programmierung
		\end{itemize}
	\item Homepage der Veranstaltung
		\begin{itemize}
			\item http://www.zib.de/de/reinefeld/courses/ws10-vl.html
			\item nach der VL
				\begin{itemize}
					\item ausführliche Folien
					\item Übungsaufgaben
					\item Ergänzendes
				\end{itemize}
		\end{itemize}
	\item Ziel
		\begin{itemize}
			\item Wie programmiert man skalierbare Programme?
		\end{itemize}
	\item Leistungsüberprüfung
		\begin{itemize}
			\item Mündliche Prüfung
			\item Termin: 23. und 24. Februar 20100
			\item Vor.:
				\begin{itemize}
					\item aktive Mitarbeit in Übung \& Bearbeitung aller Übungsaufgaben
					\item mind 67% der Übungspunkte
					\item Projektaufgabe in Gruppe gelöst \& im Plenum vorgestellt
				\end{itemize}
		\end{itemize}
\end{itemize}

\subsection{Geplanter Inhalt}

\begin{itemize}
	\item  Parallele Systeme
		\begin{itemize}
			\item Architektur: MPP, SMP (Multicore)
			\item Programmierung mit MPI und OpenMP
			\item Speedup, Effizienz
			\item Kommunikation
			\item Prozesse
			\item Namen
		\end{itemize}
		\begin{itemize}
			\item  Verteilte Systeme
			\item Abstraktion
			\item sync. Basisalgs.
			\item Unterschiede sync. \& async. Algos.
			\item Zeig \& Synchronisation
			\item Koord. und async. Basisalgos
			\item Transaktionen \& Nebenläufigskeitskontrolle
			\item Robuste Algorithmen
			\item Selbststabil. Algorthmen
			\item P2P Algorithmen
			\item Dateninsensive Anwendungen, Map/Reduce
		\end{itemize}
\end{itemize}
\subsection{Literatur}
Siehe Folien
\begin{itemize}
	\item Empfehlung: R. Guerraoui, Introduction to Reliable Distributed Programming
	\item K. Birman, Reliable Distributed Systems
\end{itemize}

\section{Einführung in Parallele Systeme}
Warum?

\begin{itemize}
	\item  Ziele
		\begin{itemize}
			\item Zeit sparen
			\item grö. Probl. lösen
		\end{itemize}
	\item  Grenzen sequentl. Rechnens \footnote{the free lunch is over (sutter):
		http://www.gotw.ca/publications/concurrency-ddj.htm}
		\begin{itemize}
			\item Lichtgeschwindigkeit: 30cm/ns im Vakuum bzw 9cm/ns in Kupfer
			\item Miniaturisierung (phys. \& ökon. Grenzen)
			\item Stromverbr. (kubisch mit Taktrate)
		\end{itemize}
	\item 
		\begin{itemize}
			\item Perf/Clock ?
		\end{itemize}
\end{itemize}

Nur Parallelität bringt zus. Leistung
\begin{itemize}
	\item  Letzt. 30 Jahre Leistungsgewinn durch
	\item 
		\begin{itemize}
			\item Taktrate
			\item Optimierung Ausführungseinheiten
			\item Cache
		\end{itemize}
	\item  Quellen für zukünft. Leistungsgewinne
	\item 
		\begin{itemize}
			\item Multicore
			\item Hyperthreading $\rightarrow$ mehrere Registerthreads, die auf mehreren Threads arbeiten $\Rightarrow$ simulierter Mehrkernprozessor, virtuelle Kerne
			\item Cache
		\end{itemize}
\end{itemize}

$\Rightarrow$ Methoden zur Unterstützung der Softwareparallelität.

\subsection{Moore's Law}
Die Anzahl der Transistoren verdoppelt sich alle zwei Jahre
$\rightarrow$ Nicht: Die Prozessorleistung!

\subsection{Reale Proz.-Leistungssteigerung}
SPEC 2006 - Abflachung um 2002

\subsection{Moore's Law heute}

\begin{itemize}
	\item Anz. Prozessorkerne verdoppelt sich mit neuer Chigeneration
		\begin{itemize}
			\item http://www.eecs.berkely.edu/Pubs/TechRpts/2008/EECS-2008-23.pdf
		\end{itemize}
	\item Warum?
		\begin{itemize}
			\item Weniger Strom durch einf. Strukturen
			\item mehr par. Speicherzugriffe $\rightarrow$ weniger Latenz
			\item Kürzere "Verkabelung" in einem Core
			\item geringere Design-Komplexität
		\end{itemize}
	\item Konsequenz
		\begin{itemize}
			\item Heute ist jeder PC Parallelrechner
			\item Hochleistungsrechner mit Mio. CPUs
		\end{itemize}
\end{itemize}


\subsection{Multicore}

\begin{itemize}
	\item  Immer mehr Multicoresysteme, z. B.
	\item 
		\begin{itemize}
			\item Intel Xeon X5570
		\end{itemize}
	\item  Eff. Nutzung schwierig
	\item  Programmierer, Compiler od. OS muss parallel ausführbare Codeseq. erkennen \& paral. Ausführung "orchestrieren"
	\item 
		\begin{itemize}
			\item Start, Sync. und Kommun. überwachen
			\item Nebenläufigkeitskontrolle durch locks
		\end{itemize}
\end{itemize}
$\Rightarrow$ Serialisierung

\section{Rechnerarchitektur}
\subsection{Komponenten einer Rechnerarchitektur}
vgl. Folie 11
\begin{itemize}
	\item Funktionales Verhalten festgel. durch
		\begin{itemize}
			\item Datenstr.
			\item Kontrollstrk.
		\end{itemize}
	\item Strukturanordnung def. durch
		\begin{itemize}
			\item Art und Anzahl HW-Betriebsmittel
			\item verbindende Kommunikationseinrichtungen
		\end{itemize}
\end{itemize}


\subsection{Operationsprinzip: Vorschft. über Zusammenwirken der Komponenten}
Strukturanord.: Struktur der Verknüpfung

\subsection{Sysarch. nach Michael Flynn (1972)}
Unterschdg. Daten- \& Befehlsströme im Rechner
vgl. Folie 13 \\
Folgefolien betrachten die einzelnen Architekturen

\subsubsection{SPMD}
Merke: SPMD (Single Program, Multiple Data) hat nichts mit Flynn Taxonomi zu tun!
\begin{itemize}
	\item Grundprinzip
		\begin{itemize}
			\item Alle (MIMD-)Prozessoren arbeiten auf Kopien desselben Programmcodes, aber mit unterschdl. Daten und ggf. in unterschiedl. Modulen
		\end{itemize}
	\item Vorteile
		\begin{itemize}
			\item Leichte Programmentwicklung, -Debugging \& -Wartung
			\item Einfachere Synchronisation (vgl. mit "echter" MIMD-Programmierung)
			\item im vgl. zu SIMD gröbere Parallelitätsganularität
		\end{itemize}
\end{itemize}


\subsubsection{Kommunikationsnetze}
Aufbau von MIMD-Systemen
Folie 23 mit Diagram (!)

\begin{itemize}
	\item SMP, DSM, nachrichtengekoppelter (shared-nothing-)Multiprozessor
\end{itemize}


\subsubsection{SMP - Multiprozessor}

\begin{itemize}
	\item glob. Speicher (UMA - uniform memory access)
	\item Glob. Adressraum
	\item Gleiche CPUs
	\item Datenaustausch
		\begin{itemize}
			\item transparentes Caching mittls Chachekohärenzprotok. in Hardware (MSI, MESI, Dragon, ...)
			\item Statusflags je Speicherseite im Cache (exclusive, invalid, shared, ...)
			\item Protokoll überwacht die Zugriffe anderer Prozessoren auf dem Bus
				\begin{itemize}
					\item schwieriger wenn kein einzelner Bus
				\end{itemize}
		\end{itemize}
\end{itemize}


\subsubsection{DSM - Multiprozessor}

\begin{itemize}
	\item Vert. Speicher (NUMA $\rightarrow$ Zugriffszeit abhängig von Position des zu erreichenden Speichers)
	\item Glob. Adressraum
	\item Gleiche CPUs (nicht zwingend relevant)
	\item Datenaustausch
		\begin{itemize}
			\item Transp. Caching mittels Cachekohärenzprot. in HW
			\item Kann auch für Cluster realisiert werden (selten)
		\end{itemize}
\end{itemize}


\subsubsection{Homogenes Multicom.-System}
\begin{itemize}
	\item vertl. Speicher
	\item Getl. Adressraum
	\item GleicheCPUs
	\item Hochleistungsnetzwerk verbindet Prozessoren
	\item Datenaustausch
		\begin{itemize}
			\item Nachrichten expl. über Netzwerk
		\end{itemize}

\end{itemize}

\subsubsection{Heterogenes Multicomp.-System}
\ldots


\section{Verteilte Systeme}
Geographisch verteiltes System

Def.: Menge miteinander verbundener, autonomer Computer, die dem Nutzer wie ein einzelnes kohärentes System erscheinen
\begin{itemize}
	\item Computer: Prozessoren/Prozesse (auch Knoten genannt)
	\item Autonom: Private Kontrolle
	\item Miteinander verbunden: Informationsaustausch
\end{itemize}

$\Rightarrow$ Leslie Lempert

\subsection{Warum?}
\begin{itemize}
	\item Austausch von Informationen
		\begin{itemize}
			\item WWW, Bittorrent,...
		\end{itemize}
	\item Erhöhte Zuverlässigkeit (reliability) durch Duplikation (replication)
	\item Gem. Verwendung von Ress. (resource sharing)
		\begin{itemize}
			\item zb. Drucker, SAN, ...
			\item Anwendung: LHC (CERN), Lofar (Astronomie)
		\end{itemize}
	\item Erhöhte Leistung (performant durch Parallelisierung)
		\begin{itemize}
			\item Google
		\end{itemize}
	\item Vereinfachung des Systemdesigns durch Spezialisierung
		\begin{itemize}
			\item spezialisierte Server
		\end{itemize}
\end{itemize}


\subsection{Typen}
... \marginpar{Das nachschlagen!}

\subsection{Forderungen}

\begin{itemize}
	\item Benutzer und Ressourcen brauchen nicht unbedingt am gl. Ort zu sein, d.h. Verteilung soll dem Benutzer verborgen sein (Transparenz)
	\item Austausch und Erweiterbarkeit von Komponenten (Offenheit)
	\item Stets gleich gute Leistung, unabhängig von Anzahl der Nutzer
\end{itemize}


\subsection{Transparenz}
Arten der Transparenz auf Folie 34 und 35

\subsection{Realisierung}
Abstraktionsschicht zw. Ress. und Anwendung $\Rightarrow$ Middleware

\subsection{Skalierbarkeit}

\begin{itemize}
	\item Verschiedene Arten der Skalierbarkeit
		\begin{itemize}
			\item Größe
			\item Geographische Verteilung
			\item Administrativ
			\item Leistung
		\end{itemize}
	\item Skalierbarkeit kann behindert werden durch
		\begin{itemize}
			\item Zentrale Komponenten
				\begin{itemize}
					\item Dienste: Einzelner Server für alle Nutzer
					\item Daten: ein einziges Online-Telefonbuch
					\item Algorithmen: Routing-Verfahren, das glob. Information braucht $\Rightarrow$ nur verteilte Algorithmen einsetzen
				\end{itemize}
			\item Kommunikation
			\item Weitere Spakete
		\end{itemize}
	\item Lösungen für Skalierbarkeit
		\begin{itemize}
			\item Verbergen von Kommunikationslatenzen durch async. statt sync. Komm.
			\item Weniger Einzelnachrichten
			\item Verteilung (Divide and Conquer)
			\item Folie 42 (!) $\Rightarrow$ Client und Server
				\begin{itemize}
					\item verbindungslose oder verbindungsorientierte Kommunikation (schnell vs. langsam, unzuverlässig vs. zuverlässig)
				\end{itemize}
			\item Mehrschichtige Architekturen (vertikale Verteilung) Folie 44
				\begin{itemize}
					\item Bsp.: Verteilte Transaktionen
				\end{itemize}
			\item Horizontale Verteilung $\Rightarrow$ P2P-Verteilung
		\end{itemize}
\end{itemize}
