\documentclass[a4paper,
12pt,
BCOR12mm,
]{scrartcl}
%scrreport
\usepackage[ngerman]{babel}
\usepackage[utf8]{inputenc}
\usepackage[T1]{fontenc}
\usepackage{url}
\title{APUVS, Blatt 2}
\author{Jan Fajerski and Kai Warncke and Magnus Müller}

\begin{document}
\maketitle  
\section*{Aufgabe 2.2}
Laut \verb|$ man mpi_send| ist die Standardmethode zum Versenden von Nachrichten in
openmpi synchron und blockierend:
\begin{verbatim}
DESCRIPTION
	MPI_Send performs a standard-mode, blocking send.
\end{verbatim}

Laut dem Testprogramm \verb|send_block| scheint es jedoch eine gepufferte Version zu sein.

\section*{Aufgabe 2.4}
Zu den Hauptproblemen zählt:
\begin{itemize}
	\item verschiedene MPI Implementationen
	\item verschiedene Rechner-Architekturen
	\item verschiedene Dateisysteme
	\item die verschiedenen Knoten sind unter Umständen nicht immer erreichbar
\end{itemize}

Lösungsansätze:
\begin{itemize}
	\item Verwendung einer einheitlichen MPI-Implementation. Hierzu gehört auch die
		Verwendung der gleichen Version, denn verschiedene Versionen sind nicht zwingend
		ABI-kompatibel.
	\item Abstraktion weg von der unterliegenden Architektur. Ähnlich der OSI-Layer sollten
	die MPI-Implementationen die zugrunde liegende Architektur möglichst verbergen.
\end{itemize}

\nocite{*}
\bibliographystyle{alphadin}
%\bibliography{bibliography.bib}

\end{document}
