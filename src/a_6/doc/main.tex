\documentclass[a4paper,
12pt,
BCOR12mm,
]{scrartcl}
%scrreport
\usepackage[ngerman]{babel}
\usepackage[utf8]{inputenc}
\usepackage[T1]{fontenc}
\usepackage{url}
\usepackage[pdftex]{graphicx}
\usepackage{listingsutf8}
\usepackage{grffile}
\usepackage{epstopdf}
\usepackage{subfigure}
\usepackage[a4paper,left=23mm,right=23mm, top=33mm, bottom=66mm]{geometry}
% lstlisting settings
\lstset{
showspaces=false,
breaklines=true,
breakindent=0pt,
frame=single,
language=C++,
extendedchars=true,
inputencoding=utf8/latin1,
identifierstyle=\ttfamily,
basicstyle=\tiny,
numbers=left,
numberstyle=\tiny,
}

\title{APUVS, Blatt 6}
\author{Jan Fajerski and Kai Warncke and Magnus Müller}

\begin{document}
% NOTE: compile with pdflatex --shell-escape main.tex

\maketitle 

\section*{Aufgabe 6.1}
Die Aufgabenstellung fragt, ob der \emph{Chandy-Lamport-Algorithmus} auch dann einen
konsistenten Schnitt erstellt, wenn er auf mehreren oder allen Prozessen gleichzeitig
gestartet wird. Dies geschieht unter der Voraussetzung, dass alle beteiligten Prozesse
bisher noch keine Markernachrichten bekommen haben. \\

Der Chandy-Lamport-Algorithmus funktioniert korrekt, auch wenn er gleichzeitig auf
mehreren Prozessen gestartet wird. \\

\begin{figure}[ht!]
  \begin{center}
    \subfigure[Ausgangssituation]{\includegraphics[scale=0.5]{graphviz/graphs/example1.pdf}}
    \subfigure[Algorithmus startet]{\includegraphics[scale=0.5]{graphviz/graphs/example2.pdf}}
    \subfigure[Marker auf alle Kanäle]{\includegraphics[scale=0.5]{graphviz/graphs/example3.pdf}}
    \subfigure[Erste Marker konsumiert]{\includegraphics[scale=0.5]{graphviz/graphs/example4.pdf}}
    \subfigure[Marker auf alle Kanäle]{\includegraphics[scale=0.5]{graphviz/graphs/example5.pdf}}
    \subfigure[Einige sind fertig]{\includegraphics[scale=0.5]{graphviz/graphs/example6.pdf}}
    \subfigure[Alle sind fertig]{\includegraphics[scale=0.5]{graphviz/graphs/example7.pdf}}
  \end{center}
  \caption{Beispiel zu Chandy-Lamport}
  \label{fig:example}
\end{figure}

\section*{Aufgabe 6.2}

\section*{Aufgabe 6.3}

\end{document}
