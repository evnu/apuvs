\documentclass[a4paper,
12pt,
BCOR12mm,
]{scrartcl}
%scrreport
\usepackage[ngerman]{babel}
\usepackage[utf8]{inputenc}
\usepackage[T1]{fontenc}
\usepackage{url}
\usepackage[pdftex]{graphicx}
\usepackage{listingsutf8}
\usepackage{grffile}
\usepackage{epstopdf}
\usepackage{subfigure}
\usepackage[a4paper,left=23mm,right=23mm, top=33mm, bottom=66mm]{geometry}
% lstlisting settings
\lstset{
showspaces=false,
breaklines=true,
breakindent=0pt,
frame=single,
language=erlang,
extendedchars=true,
inputencoding=utf8/latin1,
identifierstyle=\ttfamily,
basicstyle=\tiny,
numbers=left,
numberstyle=\tiny,
}

\usepackage{amsmath}
\usepackage{amsfonts}
\usepackage{amssymb}
% \usepackage{mathtools} not installed?
\usepackage{stmaryrd}
\usepackage{graphicx}
\usepackage[ngerman]{babel}
\usepackage{algpseudocode}

\usepackage{url}

\usepackage{color}

\usepackage{paralist} % inline list

% enhanced enumerate 
% see http://texblog.wordpress.com/2008/10/16/lists-enumerate-itemize-description-and-how-to-change-them/
\usepackage{enumerate}

% i hate the fat blob..
\renewcommand{\labelitemi}{\guilsinglright}

\usepackage[thmmarks,amsmath,amsthm]{ntheorem}

\theorempreskipamount 14pt
\theorempostskipamount 12pt
\theoremstyle{break}
\theoremheaderfont{\scshape \smallskip}
\theorembodyfont{\normalfont}
\newtheorem{defi}{Definition}[subsection]

% vektoren durch $\mat{1 \\ 2 \\ 3}$
\def\mat#1{\left(\begin{array}{cccccc}#1\end{array}\right)}

% framebox
\def\framebox#1{\fbox{\begin{minipage}{0.8\textwidth}{#1}\end{minipage}}\\}

% questionbox
\usepackage{fancybox}
\def\questionbox#1{\shadowbox{\begin{minipage}{0.8\textwidth}{ {\Huge ?} \color{red}{#1}}\end{minipage}}\\}
\def\warningbox#1{\shadowbox{\begin{minipage}{0.8\textwidth}{ {\Huge !} \color{blue}{#1}}\end{minipage}}\\}

% condensed lists
\usepackage{mdwlist}


% floor funktion
\def\floor#1{\left\lfloor #1 \right\rfloor}

\newtheorem{beh}{Behauptung}
\newtheorem{bew}{Beweis}
\newtheorem{afg}{Aufgabe}
\newtheorem{lsg}{Lösung}
\newtheorem{lem}{Lemma}[subsection]
\newtheorem{bsp}{Beispiel}[subsection]
\newtheorem{satz}{Satz}[subsection]
\newtheorem{define}{Definition}
\theoremsymbol{$\square$}

\title{APUVS, Blatt 6}
\author{Jan Fajerski and Kai Warncke and Magnus Müller}

\begin{document}
% NOTE: compile with pdflatex --shell-escape main.tex

\maketitle 

\section*{Aufgabe 6.1}
Die Aufgabenstellung fragt, ob der \emph{Chandy-Lamport-Algorithmus} auch dann einen
konsistenten Schnitt erstellt, wenn er auf mehreren oder allen Prozessen gleichzeitig
gestartet wird. Dies geschieht unter der Voraussetzung, dass alle beteiligten Prozesse
bisher noch keine Markernachrichten bekommen haben. \\

Der Chandy-Lamport-Algorithmus funktioniert korrekt, auch wenn er gleichzeitig auf
mehreren Prozessen gestartet wird. \\

Wir illustrieren das an dem durch Abbildung \ref{fig:example} gegebenen Beispiel. Wir
beginnen in einem Zustand, der durch Abbildung \ref{fig:subex1} gegeben ist: Der
Prozessgraph ist zusammenhängend und besteht aus 4 Prozessen und 4 Kommunikationskanälen.
Nun wird der Chandy-Lamport-Algorithmus zeitgleich (zumindest ohne zu große Verzögerung
dazwischen) auf den Prozessen 1 und 4 gestartet (vgl. Abbildung \ref{fig:subex2}). Der
Start des Algorithmus funktioniert auch über Marker, welche diese beiden Prozesse bereits
erhalten haben (nicht dargestellt). Diese aktivierten Prozesse senden an alle
Kommunikationskanäle Marker, nachdem sie ihren internen Zustand gesichert haben (Abbildung
\ref{fig:subex3}). Zudem starten sie danach die Aufzeichnung der eingehenden Nachrichten
über die angeschlossenen Kommunikationskanäle. Nun folgt die in Abbildung \ref{fig:subex4}
dargestelle Situation: Die Prozesse 2 und 3 erhalten ihrerseits ihre ersten Marker und
beginnn somit, den Algorithmus auszuführen. Sie sichern also ihren internen Zustand und
senden dann, wie in Abbildung \ref{fig:subex5} dargestellt, auch Marker an alle
angeschlossenen Kommunikationskanäle. Wie an Abbildung \ref{fig:subex5} zu sehen ist,
wurden aber noch nicht alle im System vorhanden Markernachrichten konsumiert. Dies ist
eine wichtige Erkenntnis, denn dadurch sind die auf diesem Kanal benötigten
Markernachrichten bereits vorhanden. Abbildung \ref{fig:subex6} beschreibt einen
Ausschnitt aus dem System, nachdem die Prozesse 2 und 3 ihre Marker geschickt haben, diese
aber noch nicht konsumiert wurden. Nun sind die für die Prozesse 1 und 4 nötigen Marker
unterwegs und können konsumiert werden. Dadurch kommen 1 und 4 in den Endzustand und
können den Algorithmus beenden. Die Prozesse 2 und 3 warten wiederrum noch auf eingehende
Nachrichten. Wie an den Kanälen noch zu sehen, befinden sich immer noch Markernachrichten
in den Puffern. Aufgrund der FIFO-Eigenschaft der Kanäle können diese Nachrichten nicht
überholt werden. Das bedeutet, dass auch neue Nachrichten von den bereits mit dem
Algorithmus fertigen Prozessen diese Marker nicht überholen werden. Somit werden die
Marker später, wie in Abbildung \ref{fig:subex7} dargestellt,
von den Prozessen 2 und 3 konsumiert und diese können wiederrum in den Endzustand
übergehen.

\subsection*{Schlussfolgerung}
Der Algorithmus ist also stabil gegenüber mehrfachem Start auf unterschiedlichen
Prozessen, da die Marker nicht überholt werden können (FIFO-Eigenschaft). Es besteht also
nicht die Gefahr, dass Nachrichten mit in den Schnitt aufgenommen werden, die zu spät
abgesendet wurden -- also solche Nachrichten, deren Sender bereits den Algorithmus
abgeschlossen hat. In unserem Beispiel könnte der Prozess 1 in Abildung \ref{fig:subex6} bereits neue Nachrichten an
Prozess 2 schicken, ohne dass dies den Schnitt kaputt machen würde, denn Prozess 2 wartet
noch auf einen Marker und erhält diesen wegen der FIFO-Eigenschaft vor der neuen Nachricht
von Prozess 1. \\
Damit besteht die \emph{Grenze des Schnitts} (vergleiche Folie 35, Vorlesung 6) auch
weiterhin nur aus dem Empfang von Markern auf allen eingehenden Kommunikationskanälen.

\begin{figure}[ht!]
  \begin{center}
    \subfigure[Ausgangssituation]      {\label{fig:subex1} \includegraphics[scale=0.5]{graphviz/graphs/example1.pdf}}
    \subfigure[Algorithmus startet]    {\label{fig:subex2} \includegraphics[scale=0.5]{graphviz/graphs/example2.pdf}}
    \subfigure[Marker auf alle Kanäle] {\label{fig:subex3} \includegraphics[scale=0.5]{graphviz/graphs/example3.pdf}}
    \subfigure[Erste Marker konsumiert]{\label{fig:subex4} \includegraphics[scale=0.5]{graphviz/graphs/example4.pdf}}
    \subfigure[Marker auf alle Kanäle] {\label{fig:subex5} \includegraphics[scale=0.5]{graphviz/graphs/example5.pdf}}
    \subfigure[Einige sind fertig]     {\label{fig:subex6} \includegraphics[scale=0.5]{graphviz/graphs/example6.pdf}}
    \subfigure[Alle sind fertig]       {\label{fig:subex7} \includegraphics[scale=0.5]{graphviz/graphs/example7.pdf}}
  \end{center}
  \caption{Beispiel zu Chandy-Lamport}
  \label{fig:example}
\end{figure}

\section*{Aufgabe 6.2}
Siehe Anhang auf Seite \pageref{Anhang}. \verb|lamdy.erl| enthält das Beispiel aus der
Vorlesung, \verb|snapshot.erl| die Implementation des Snapshotalgorithmus.
\paragraph{Anmerkungen}
\begin{itemize}
  \item Der Zustand der Prozesse muss vor Betreten der Snapshotfunktion
    gespeichert werden. Wir haben uns hierfür entschieden, da eine Abstraktion der
    Zustandsspeicherung nicht sinnvoll ist.
  \item Die Snapshotfunktion merkt sich alle empfangenen Nachrichten einer Liste. Ein
    aufrufender Prozess muss diese Liste nach Erstellung des Schnitts selbstständig wieder
    abarbeiten. 
  \item Der initiale Marker wird im Beispiel durch den \verb|distributor| versendet, falls
    dessen Nachrichtenpuffer leer ist.
\end{itemize}
\section*{Aufgabe 6.3}
\begin{beh}
  Für zwei Ereignisse $e,e'$ gilt: \[e \rightarrow e' \Longleftrightarrow V(e) < V(e')\]
\end{beh}
\begin{bew}
  "`$\Longrightarrow$"'. Voraussetzung: $e \rightarrow e'$. Zu zeigen: $V(e) < V(e')$. \\
  Wir zeigen dies per Induktion über die Eigenschaften der \emph{Happens-Before
  Relation}. \\
\textbf{Induktionsanfang}. \\

Gelte $e \rightarrow e'$ wegen (HB1), also $e \rightarrow_i e'$. Offensichtlich gilt,
genau wie bei der Lamportuhr, dass $V_i(e) < V_i(e')$ ist. Es gilt also noch zu zeigen,
dass ebenso $V_j(e) \leq V_j(e'), \forall j \neq i$ zutrifft. Nehmen wir an, dies würde
nicht zutreffen. Dann muss der entsprechende Prozess $j$, der diese Bedingung nicht
erfüllt, mit Prozess $i$ "`synchronisiert"' worden sein\footnote{Diese zwei Prozesse
müssen also einmal in der Zwischenzeit ihre Zeitstempel über \emph{send-receive} angepasst
haben.} Das Ereignis $e$ muss nach Voraussetzung vor Ereignis $e'$ in $i$ passiert sein.
Wenn nun aber $V_j(e) > V_j(e')$ gilt, dann muss die Synchronisation von $i$ und $j$
nach $e'$ passiert sein, jedoch vor $e$. $e'$ kann also nicht vor $e$ passiert sein, was
im Widerspruch zur Voraussetzung steht.\\

Sei nun $e = send(m), e' = receive(m,t)$, wobei $m$ eine Nachricht und $t$ der Zeitstempel
bei Versand von $M$ sind. Es gilt also (HB2). Dann folgt offensichtlich $V(e) < V(e')$, da $V(e')$ als
komponentenweises Maximum von $V(e)$ und dem im Empfängerprozess $j$ aktuellen Vektor $V'$
gebildet wird. In $V'$ wird vor Empfang der Nachricht $V_j$ inkrementiert, weshalb $V(e)
\neq V(e')$ sein muss.

\textbf{Induktionsschritt}
Es gelte $e \rightarrow e' \rightarrow e''$, also (HB2). Nach Induktionsvoraussetzung
gilt $V(e) < V(e')$ und $V(e') < V(e'')$. Da die Relation $<$ aber transitiv ist folgt
direkt $V(e) < V(e'')$, was zu zeigen war. \\
Somit ist die Vorwärtsrichtung erfolgreich bewiesen.

\end{bew}
\begin{bew}
  "`$\Longleftarrow$"'. Voraussetzung: $V(e) < V(e')$. Zu zeigen: $e \rightarrow e'$. \\
  Falls $e$ und $e'$ Ereignisse des gleichen Prozesses $i$ sind ist dies offensichtlich.
  Ebenso gilt dies offensichtlich, falls $e = send\ m$ und $e' = receive\ m$ ist. Zudem gilt
  für $e,e'$, dass $e''$ existiert, so dass $V(e) < V(e'') <= V(e')$. Auf dieser
  Grundlage kann man nun folgendermaßen einen gerichteten Graph $G$ mit folgenden
  Eigenschaften erstellen: Zwei beliebige Ereignisse $e_n,e_m$ (oder auch Knoten) werden
  durch eine Kante $(e_n,e_m)$ verbunden, falls 
  \begin{enumerate}[(a)]
    \item Beide Ereignisse des gleichen Prozess $i$ sind und $V_i(e_n) + 1 = V_i(e_m)$
      (also direkt aufeinander folgende Ereignisse)
    \item Falls $e_n = send\ message$ und $e_m = receive\ message$ für eine Nachricht.
  \end{enumerate}
  Der so konstruierte Graph ist also konsistent mit der \emph{happens before} Relation,
  denn zwei Ereignisse sind nur direkt verbunden, falls das eine Ereignis vor dem anderen
  passiert. \\
  Für diesen Graph gilt nun aber folgende Eigenschaft: Falls für zwei Ereignisse $e$ und $e'$
  die Voraussetzung $V(e) < V(e')$ gilt, dann existiert ein Weg $w = (e, \dots, e_a, \dots e_b, \dots, e')$ innerhalb des Graphen
  von $e$ nach $e'$. Für zwei Ereignisse $e_a, e_b$ des Weges $w$ gilt aber $e_a \rightarrow e_b$, weshalb auch $e \rightarrow e'$ folgt.

\end{bew}


\pagebreak
\section*{Anhang}
\label{Anhang}
\begin{appendix}
 \lstinputlisting{../src/lamdy.erl} 
 \lstinputlisting{../src/snapshot.erl} 
\end{appendix}

\end{document}
