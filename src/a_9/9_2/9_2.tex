Die miniale Zeit T' ist in Abhängigkeit von i bei N Prozessen wie folgt definiert:
\begin{eqnarray*}
  T'(N) & = & T \\
  T'(i) & = & T'(i+1) + T_{PROCESS} + T_{TRANS}
\end{eqnarray*}
Der höchste Prozess (also derjenige mit der höchsten Prozessnummer) wird nie die
den Ausführungszweig in dem die Zeit T' gewartet wird abarbeiten, da er nie eine
Antwort auf seine e-Nachricht bekommt. Darum kann man T' des höchsten Prozesses
einfach auf T setzen. \\
Der darunterliegende Prozess muss mindestens diese Zeit, plus die Zeit die
Nächsthöhere zum Versenden der eigenen e-Nachricht benötigt ($T_{PROCESS}$) und
der Nachrichtenlaufzeit der c-Nachricht benötigt, warten.
