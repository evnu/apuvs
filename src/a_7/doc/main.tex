\documentclass[a4paper,
12pt,
BCOR12mm,
]{scrartcl}
%scrreport
\usepackage[ngerman]{babel}
\usepackage[utf8]{inputenc}
\usepackage[T1]{fontenc}
\usepackage{url}
\usepackage[pdftex]{graphicx}
\usepackage{listingsutf8}
\usepackage{grffile}
\usepackage{epstopdf}
\usepackage{subfigure}
\usepackage[a4paper,left=23mm,right=23mm, top=33mm, bottom=66mm]{geometry}
% lstlisting settings
\lstset{
showspaces=false,
breaklines=true,
breakindent=0pt,
frame=single,
language=erlang,
extendedchars=true,
inputencoding=utf8/latin1,
identifierstyle=\ttfamily,
basicstyle=\tiny,
numbers=left,
numberstyle=\tiny,
}

\usepackage{amsmath}
\usepackage{amsfonts}
\usepackage{amssymb}
% \usepackage{mathtools} not installed?
\usepackage{stmaryrd}
\usepackage{graphicx}
\usepackage[ngerman]{babel}
\usepackage{algpseudocode}

\usepackage{url}

\usepackage{color}

\usepackage{paralist} % inline list

% enhanced enumerate 
% see http://texblog.wordpress.com/2008/10/16/lists-enumerate-itemize-description-and-how-to-change-them/
\usepackage{enumerate}

% i hate the fat blob..
\renewcommand{\labelitemi}{\guilsinglright}

\usepackage[thmmarks,amsmath,amsthm]{ntheorem}

\theorempreskipamount 14pt
\theorempostskipamount 12pt
\theoremstyle{break}
\theoremheaderfont{\scshape \smallskip}
\theorembodyfont{\normalfont}
\newtheorem{defi}{Definition}[subsection]

% vektoren durch $\mat{1 \\ 2 \\ 3}$
\def\mat#1{\left(\begin{array}{cccccc}#1\end{array}\right)}

% framebox
\def\framebox#1{\fbox{\begin{minipage}{0.8\textwidth}{#1}\end{minipage}}\\}

% questionbox
\usepackage{fancybox}
\def\questionbox#1{\shadowbox{\begin{minipage}{0.8\textwidth}{ {\Huge ?} \color{red}{#1}}\end{minipage}}\\}
\def\warningbox#1{\shadowbox{\begin{minipage}{0.8\textwidth}{ {\Huge !} \color{blue}{#1}}\end{minipage}}\\}

% condensed lists
\usepackage{mdwlist}


% floor funktion
\def\floor#1{\left\lfloor #1 \right\rfloor}

\newtheorem{beh}{Behauptung}
\newtheorem{bew}{Beweis}
\newtheorem{afg}{Aufgabe}
\newtheorem{lsg}{Lösung}
\newtheorem{lem}{Lemma}[subsection]
\newtheorem{bsp}{Beispiel}[subsection]
\newtheorem{satz}{Satz}[subsection]
\newtheorem{define}{Definition}
\theoremsymbol{$\square$}

\title{APUVS, Blatt 7}
\author{Jan Fajerski and Kai Warncke and Magnus Müller}

\begin{document}
% NOTE: compile with pdflatex --shell-escape main.tex

\maketitle 

\section*{Aufgabe 7.1}
Bei der Implementierung eines elektronischen Aktienhandels bietet sich
\emph{Entry-Konsisten} an. Da im Aktienhandel sehr viele Aktien zu jedem Zeitpunkt
gehandelt werden, ist Effizienz der verwendeten Algorithmen sehr wichtig. Entry-Konsistenz
bietet gegenüber \emph{Release-Konsistenz} den Vorteil, dass nur die Daten synchronisiert
werden müssen, die wirklich verwendet werden. \emph{Schwache Konsistenz} ist hingegen
nicht ausreichend, um eine erfolgreiche Synchronisierung durchzuführen. Da eine Aktie in
einer $1:1$-Beziehung mit einem Besitzer steht, dürfen sich Zugriffe nicht überlappen.
Damit dürfen auch Schreibzugriffe nicht beliebig gemischt werden, sondern müssen
synchronisiert werden.\footnote{\emph{Lokale} Aktien, die dann synchronisiert werden,
würden Rollbackstrategien erfordern, falls es zu Konflikten kommt.}
\section*{Aufgabe 7.2}
\subsection*{a)}
\begin{table}
  \centering
  \begin{tabular}{|cccccccc}
      \hline
      P1: & W(a)0 & $\rightarrow$ & W(a)1 && && \\
      \hline
      P2: & && R(a)1 & $\rightarrow$ & W(b)2 && \\
      \hline
      P3: & && && R(b)2 & & R(a)0 \\
      \hline
  \end{tabular}
  \caption{Mögliche Ausführungsreihenfolge}
  \label{tab:72a}
\end{table}
Wir gehen davon aus, dass die Variablen mit 0 initialisiert werden.\footnote{Wenn die
Initialbelegung beliebig gewählt werden kann, dann kann auch eine kausal konsistente
Reihenfolge gebildet werden.} Dann ergibt sich zum
Beispiel die Ausführungsreihenfolge in Abbildung \ref{tab:72a}, wobei die Pfeile $\rightarrow$ die kausalen
Zusammenhänge darstellen. Diese ergeben sich durch folgende Betrachtung: Innerhalb eines
Prozesses gilt sowieso die Ausführungsreihenfolge. Nun gilt, dass P2 aus \verb|a| eine
\verb|1| liest. Damit muss dieser Lesevorgang aber nach dem Schreibvorgang $W(a)1$ durch
P1 erfolgt sein. $W(b)2$ in P2 muss wegen Ausführungsreihenfolge innerhalb eines Prozesses
nach dem Schreibvorgang erfolgen. Da $b$ zu Anfang mit $0$ initialisiert wird und erst
durch den Schreibvorgang $W(b)2$ auf 2 gesetzt wird, muss der Lesevorgang $R(b)2$ von P3
nach dem Schreibvorgang $W(b)2$ von P2 erfolgen. Nun gilt aber immernoch, dass die
Variable \verb|a| mit \verb|1| belegt ist. Somit kann P3 nicht $R(a)0$ lesen.
\subsection*{b)}
Die gegebene Ausführung ist nicht sequentiell konsistent. Nehmen wir an, dass die
Variablen mit 0 initialisiert sind. Dann folgt, dass P1 erst dann $R(x)1$ lesen kann, wenn
P2 die Schreiboperation $W(x)1$ durchgeführt hat. Damit nun aber P1 $R(x)2$ lesen kann,
muss zuerst die Schreiboperation $W(x)2$ durch P2 durchgeführt werden. Bevor dies
passiert, muss P2 zuerst die Leseoperation $R(y)1$ durchführen. Da \verb|y| zu Beginn noch
mit 0 initialisiert ist, muss P1 erst $W(y)1$ durchführen. Dies widerspricht aber der
Reihenfolge der Ausführung innerhalb des Prozesses, denn damit müsste $W(y)1$ vor $R(x)2$
in P1 durchgeführt werden.
\subsection*{c)}
{\color{red}{Muss noch erledigt werden.}}

\section*{Aufgabe 7.3}
 \lstinputlisting{../src/com.erl} 
 \lstinputlisting{../src/bem.erl} 
\end{document}
